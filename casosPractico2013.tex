\section{Caso práctico 2014}

Desarrollo del software asociado a un cajero automático de un banco genérico, llamémosle \textbf{RABENSO}.

\subsection{Detalla formalmente las siguientes funcionalidades, según el modelo empleado y la plantilla que empleasteis en las prácticas}
\begin{itemize}
    \item RF1: Cambiar PIN.
    \item RF2: Consultar movimientos anteriores.
    \item RF3: Retirar dinero.
\end{itemize}

\begin{itemize}
    \item \textbf{ID:} RF-0001
    \item \textbf{Título:} Cambiar pin.
    \item \textbf{Descripción:} el sistema deberá permitir cambiar el número de identificación personal del usuario utilizado para autorizar transacciones bancarias. Para ello utilizará una interfaz gráfica con un teclado numérico virtual en pantalla y confirmación de pin anterior. Si el número anterior no es correcto no se realizará ningún cambio de PIN.
    \item \textbf{Importancia:} Vital.
    \item \textbf{Estabilidad:} Alta.
    \item \textbf{Fuente:} Julián Flores González.
    \item \textbf{Criterio de validación}: el sistema cambia el número de identificación según las directrices anteriormente especificadas.
\end{itemize}

\begin{itemize}
    \item \textbf{ID:} RF-0002
    \item \textbf{Título:} Consultar movimientos.
    \item \textbf{Descripción:} el sistema deberá mostrar todos los movimientos del usuario en una lista desplegable por pantalla. Deberá permitir filtrar los movimientos por fecha (es decir, especificar un rango de fechas para visualizar).
    \item \textbf{Importancia:} Vital.
    \item \textbf{Estabilidad:} Alta.
    \item \textbf{Fuente:} Julián Flores González.
    \item \textbf{Criterio de validación}: El sistema muestra el conjunto de movimientos correctamente.
\end{itemize}

\begin{itemize}
    \item \textbf{ID:} RF-0003
    \item \textbf{Título:} Retirar dinero.
    \item \textbf{Descripción:} el sistema deberá retirar dinero de la cuenta bancaria del usuario. Para ello, se sustrae la cantidad especificada por el usuario en el teclado numérico virtual habilitado para tal propósito y se dispone un conjunto de billetes con dicha cantidad (minimizando el número de billetes emitidos).
    \item \textbf{Importancia:} Vital.
    \item \textbf{Estabilidad:} Alta.
    \item \textbf{Fuente:} Julián Flores González.
    \item \textbf{Criterio de validación}: el sistema realiza las operaciones correctamente (sustracción y emisión de billetes).
\end{itemize}

\subsection{Diseña el modelo de contexto, el DFD de primer nivel de forma que solo englobe las funcionalidades del apartado anterior}

\subsection{Para el RF3, haz el DFD de nivel 2, con su correspondiente DFC asociado}

\subsection{En el siguiente contexto, ¿cuál sería el modelo de ciclo de vida más apropiado? Motiva tu respuesta:}
``El software va a ser desarrollado por una empresa que amplia experiencia en el desarrollo software bancario. Los requisitos están perfectamente definidos, pero el cliente quiere dotar al cajero de funcionalidades adicionales a las clásicas que le darían una ventaja competitiva sobre la competencia (que es muy activa), lo que exige resultados operativos a corto plazo.''
%También podría ser espiral por lo mismo que en 2014
El ciclo de vida elegido es ``Programación Extrema'' por los siguientes motivos:
\begin{itemize}
    \item La dinámica de historias de usuario permite al cliente priorizar las funcionalidades que quiera por lo que es útil si se requieren resultados a corto plazo.
    \item La dinámica de historias de usuario proporciona entregables frecuentes lo que posibilita tener resultados operativos a corto plazo.
    \item Al ser una metodología ágil, tiene una menor carga de modelos y documentación en comparación con las metodologías clásicas por lo que las entregas serán más rápidas.
\end{itemize}

\subsection{Establece un proceso de administración de riesgos, siguiendo la metodología de Sommerville, que contemple solo 3 riesgos, para el desarrollo del software en el siguiente contexto:}
``El cajero tiene que admitir tarjetas de los siguientes tipos: RB, Rabocard, ENSO, TaboaBank. Tiene que tener un funcionamiento 24/7. Sabemos que en el mercado hay competidores que están trabajando en un producto similar''

\subsubsection{Identificación y análisis de riesgos}
%Aquí lo voy a realizar por separado
\begin{itemize}
    \item \textbf{ID}: R-001
    \item \textbf{Nombre}: Aparece un nuevo tipo de tarjeta de crédito.
    \item \textbf{Descripción}:
\end{itemize}

\begin{itemize}
    \item \textbf{ID}: R-002
    \item \textbf{Nombre}: el cajero deja de dar servicio debido a un error de software.
    \item \textbf{Descripción}: Un error de software provoca que el cajero quede inutilizado.
\end{itemize}

\begin{itemize}
    \item \textbf{ID}: R-003
    \item \textbf{Nombre}: Un competidor crea un producto similar que supera en funcionalidades nuestro producto.
\end{itemize}

\subsubsection{Planificación de riesgos}


\subsection{Pensando en el proceso de pruebas, establece las clases de equivalencia para las siguientes entradas:}
\begin{itemize}
    \item Tipo de tarjeta
    \item PIN
    \item ¿Desea imprimir el recibo?
\end{itemize}